
\documentclass{beamer}
\usepackage[utf8]{inputenc}
\usepackage{listings}
\usepackage{xcolor}

\definecolor{aquagreen}{RGB}{200,255,240}
\setbeamercolor{block body alerted}{bg=aquagreen}

\title{Kinetic Monte Carlo Simulation}
\author{Generated by ChatGPT}
\date{\today}

\begin{document}

\begin{frame}[fragile]
  \titlepage
\end{frame}

\begin{frame}[fragile]
  \frametitle{Overview}
  This presentation describes the structure and implementation of a basic Kinetic Monte Carlo (KMC) simulation using C++.
  
  We are going to simulate the simple Langevin SDE 
  
  \[
  \mathrm{d}\mathbf{x}_{i} = \left( \dfrac{D}{k_{B}T} \displaystyle\sum_{i}\mathbf{F}_{ij} + \mathbf{u}_{f}\right) +
  \left(\Delta x \right)_{i}^{B}
  \]
  
  In a 2D channel of hight $h$ filled with a fluid of velocity $\mathbf{u}_{f}$.
  \begin{itemize}
\item  $\mathbf{F}_{ij}$ denotes the $j$-th force acting on the $i$-th particle.
\item  $\mathbf{x}_{i}$ denodes the 2-component vector of particle coordinates.
\item  $\left(\Delta x \right)_{i}^{B}$ are independent Gaussian process to represent Brownian motion (diffusion).
  \end{itemize}
\end{frame}

\begin{frame}[fragile]
  \frametitle{Loop over timesteps}
  
\begin{minipage}{\linewidth}
\begin{minipage}{0.97\linewidth}
\begin{lstlisting}[language=C++]
  for (int it = 0; it<Nt; ++it){    
\end{lstlisting}
\end{minipage}
\end{minipage}

\end{frame}

\begin{frame}[fragile]
  \frametitle{Output (if any)}
  
\begin{minipage}{\linewidth}
\begin{minipage}{0.97\linewidth}
\begin{lstlisting}[language=C++]
  std::ofstream outFile;
  std::cout << "Simulation complete. Data saved to particle_positions.csv.\n"; 
\end{lstlisting}
\end{minipage}
\end{minipage}

\end{frame}

\begin{frame}[fragile]
  \frametitle{Compilation Instructions}
  \begin{itemize}
    \item Compile the code using C++17:
    \begin{lstlisting}[language=bash]
make
    \end{lstlisting}
    \item Clean build files:
    \begin{lstlisting}[language=bash]
make clean
    \end{lstlisting}
  \end{itemize}
\end{frame}

\end{document}
